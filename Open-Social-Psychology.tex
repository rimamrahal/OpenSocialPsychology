% Options for packages loaded elsewhere
\PassOptionsToPackage{unicode}{hyperref}
\PassOptionsToPackage{hyphens}{url}
\PassOptionsToPackage{dvipsnames,svgnames,x11names}{xcolor}
%
\documentclass[
  letterpaper,
]{book}

\usepackage{amsmath,amssymb}
\usepackage{iftex}
\ifPDFTeX
  \usepackage[T1]{fontenc}
  \usepackage[utf8]{inputenc}
  \usepackage{textcomp} % provide euro and other symbols
\else % if luatex or xetex
  \usepackage{unicode-math}
  \defaultfontfeatures{Scale=MatchLowercase}
  \defaultfontfeatures[\rmfamily]{Ligatures=TeX,Scale=1}
\fi
\usepackage{lmodern}
\ifPDFTeX\else  
    % xetex/luatex font selection
\fi
% Use upquote if available, for straight quotes in verbatim environments
\IfFileExists{upquote.sty}{\usepackage{upquote}}{}
\IfFileExists{microtype.sty}{% use microtype if available
  \usepackage[]{microtype}
  \UseMicrotypeSet[protrusion]{basicmath} % disable protrusion for tt fonts
}{}
\makeatletter
\@ifundefined{KOMAClassName}{% if non-KOMA class
  \IfFileExists{parskip.sty}{%
    \usepackage{parskip}
  }{% else
    \setlength{\parindent}{0pt}
    \setlength{\parskip}{6pt plus 2pt minus 1pt}}
}{% if KOMA class
  \KOMAoptions{parskip=half}}
\makeatother
\usepackage{xcolor}
\setlength{\emergencystretch}{3em} % prevent overfull lines
\setcounter{secnumdepth}{5}
% Make \paragraph and \subparagraph free-standing
\ifx\paragraph\undefined\else
  \let\oldparagraph\paragraph
  \renewcommand{\paragraph}[1]{\oldparagraph{#1}\mbox{}}
\fi
\ifx\subparagraph\undefined\else
  \let\oldsubparagraph\subparagraph
  \renewcommand{\subparagraph}[1]{\oldsubparagraph{#1}\mbox{}}
\fi


\providecommand{\tightlist}{%
  \setlength{\itemsep}{0pt}\setlength{\parskip}{0pt}}\usepackage{longtable,booktabs,array}
\usepackage{calc} % for calculating minipage widths
% Correct order of tables after \paragraph or \subparagraph
\usepackage{etoolbox}
\makeatletter
\patchcmd\longtable{\par}{\if@noskipsec\mbox{}\fi\par}{}{}
\makeatother
% Allow footnotes in longtable head/foot
\IfFileExists{footnotehyper.sty}{\usepackage{footnotehyper}}{\usepackage{footnote}}
\makesavenoteenv{longtable}
\usepackage{graphicx}
\makeatletter
\def\maxwidth{\ifdim\Gin@nat@width>\linewidth\linewidth\else\Gin@nat@width\fi}
\def\maxheight{\ifdim\Gin@nat@height>\textheight\textheight\else\Gin@nat@height\fi}
\makeatother
% Scale images if necessary, so that they will not overflow the page
% margins by default, and it is still possible to overwrite the defaults
% using explicit options in \includegraphics[width, height, ...]{}
\setkeys{Gin}{width=\maxwidth,height=\maxheight,keepaspectratio}
% Set default figure placement to htbp
\makeatletter
\def\fps@figure{htbp}
\makeatother
% definitions for citeproc citations
\NewDocumentCommand\citeproctext{}{}
\NewDocumentCommand\citeproc{mm}{%
  \begingroup\def\citeproctext{#2}\cite{#1}\endgroup}
\makeatletter
 % allow citations to break across lines
 \let\@cite@ofmt\@firstofone
 % avoid brackets around text for \cite:
 \def\@biblabel#1{}
 \def\@cite#1#2{{#1\if@tempswa , #2\fi}}
\makeatother
\newlength{\cslhangindent}
\setlength{\cslhangindent}{1.5em}
\newlength{\csllabelwidth}
\setlength{\csllabelwidth}{3em}
\newenvironment{CSLReferences}[2] % #1 hanging-indent, #2 entry-spacing
 {\begin{list}{}{%
  \setlength{\itemindent}{0pt}
  \setlength{\leftmargin}{0pt}
  \setlength{\parsep}{0pt}
  % turn on hanging indent if param 1 is 1
  \ifodd #1
   \setlength{\leftmargin}{\cslhangindent}
   \setlength{\itemindent}{-1\cslhangindent}
  \fi
  % set entry spacing
  \setlength{\itemsep}{#2\baselineskip}}}
 {\end{list}}
\usepackage{calc}
\newcommand{\CSLBlock}[1]{\hfill\break\parbox[t]{\linewidth}{\strut\ignorespaces#1\strut}}
\newcommand{\CSLLeftMargin}[1]{\parbox[t]{\csllabelwidth}{\strut#1\strut}}
\newcommand{\CSLRightInline}[1]{\parbox[t]{\linewidth - \csllabelwidth}{\strut#1\strut}}
\newcommand{\CSLIndent}[1]{\hspace{\cslhangindent}#1}

\usepackage{fancyhdr}
\usepackage{graphicx}
\usepackage{eso-pic}
\usepackage{tikz}
\AtBeginDocument{\thispagestyle{empty}\begin{tikzpicture}[remember picture,overlay] \node at (current page.center) [yshift=1cm] {\includegraphics[width=0.75\paperwidth,height=0.9\paperheight,keepaspectratio]{cover.png}}; \end{tikzpicture}\clearpage}
\makeatletter
\@ifpackageloaded{bookmark}{}{\usepackage{bookmark}}
\makeatother
\makeatletter
\@ifpackageloaded{caption}{}{\usepackage{caption}}
\AtBeginDocument{%
\ifdefined\contentsname
  \renewcommand*\contentsname{Table of contents}
\else
  \newcommand\contentsname{Table of contents}
\fi
\ifdefined\listfigurename
  \renewcommand*\listfigurename{List of Figures}
\else
  \newcommand\listfigurename{List of Figures}
\fi
\ifdefined\listtablename
  \renewcommand*\listtablename{List of Tables}
\else
  \newcommand\listtablename{List of Tables}
\fi
\ifdefined\figurename
  \renewcommand*\figurename{Figure}
\else
  \newcommand\figurename{Figure}
\fi
\ifdefined\tablename
  \renewcommand*\tablename{Table}
\else
  \newcommand\tablename{Table}
\fi
}
\@ifpackageloaded{float}{}{\usepackage{float}}
\floatstyle{ruled}
\@ifundefined{c@chapter}{\newfloat{codelisting}{h}{lop}}{\newfloat{codelisting}{h}{lop}[chapter]}
\floatname{codelisting}{Listing}
\newcommand*\listoflistings{\listof{codelisting}{List of Listings}}
\makeatother
\makeatletter
\makeatother
\makeatletter
\@ifpackageloaded{caption}{}{\usepackage{caption}}
\@ifpackageloaded{subcaption}{}{\usepackage{subcaption}}
\makeatother
\ifLuaTeX
  \usepackage{selnolig}  % disable illegal ligatures
\fi
\usepackage{bookmark}

\IfFileExists{xurl.sty}{\usepackage{xurl}}{} % add URL line breaks if available
\urlstyle{same} % disable monospaced font for URLs
\hypersetup{
  pdftitle={Open Social Psychology},
  pdfauthor={Rima-Maria Rahal},
  colorlinks=true,
  linkcolor={green},
  filecolor={Maroon},
  citecolor={pink},
  urlcolor={Blue},
  pdfcreator={LaTeX via pandoc}}

\title{Open Social Psychology}
\author{Rima-Maria Rahal}
\date{2024-07-27}

\begin{document}
\frontmatter
\maketitle

\renewcommand*\contentsname{Table of contents}
{
\hypersetup{linkcolor=}
\setcounter{tocdepth}{2}
\tableofcontents
}
\mainmatter
\bookmarksetup{startatroot}

\chapter*{\texorpdfstring{{Preface}}{Preface}}\label{preface}
\addcontentsline{toc}{chapter}{{Preface}}

\markboth{{Preface}}{{Preface}}

Social psychology is devoted to studying how individuals behave, think
and feel within their social contexts. The field is therefore, by its
very nature, set up for collaborative work. Leveraging the social
context in which knowledge is generated is built in to the assumptions
and interests that social psychology pursues. This fundamental attitude
towards social embeddedness of knowledge is mirrored in the process by
which this study book came to be. Through bringing together the work of
students at Heidelberg University during the winter term of 2023, this
volume offers diverse perspectives on a shared target topic: changing
perceptions of classical social psychological research.

Social psychology lore is built on a strong set of classical research
paradigms and findings, featured in many of the textbooks, syllabi,
online courses and teaching guides that aspiring psychologists study
with and established psychologists use as teaching resources. However,
the common body of knowledge that social psychology relies on is
undergoing change. Modern research methods and changing attitudes
towards permissible research practices bring about social psychological
research that looks much different today than it used. This book is
dedicated to tracing some of these changes, and to offering a version of
record of the changing perceptions and interpretations of classic social
psychology in the light of it's contemporary counterpart. As such, this
study book is a snapshot of how we see social psychology today.

Because it tends to be difficult to keep teaching and study materials up
to date with emerging trends and debates, we see this study book as an
addition to traditional educational resources in social psychology,
published as an Open Educational Resource to aid the accessibility of
this knowledge for all, and to be adapted to teachers' and learners'
needs as they dive into what social psychology has to offer.

\bookmarksetup{startatroot}

\chapter*{\texorpdfstring{{How to Use this
Book}}{How to Use this Book}}\label{how-to-use-this-book}
\addcontentsline{toc}{chapter}{{How to Use this Book}}

\markboth{{How to Use this Book}}{{How to Use this Book}}

This book contains several types of resources: narrative text,
definitions and questions for reflection, as well as references.

In XX chapters, we provide narrative summaries about classical research
in social psychology and its modern follow-up. Often, this means we
include new attempts to show the same finding (replication attempts) or
meta-analytical work that brings together a lot of evidence from
different sources regarding a certain hypothesis.

Because this work is targeted at students, we provide definitions of key
terms, preceded by {\#definition} and displayed like this:

\begin{quote}
{\#definition} Replication

An attempt to find the same result as a previous study in a new data
set.
\end{quote}

We have aimed at providing a critical but neutral perspective to the
classical and modern studies of social psychology discussed in the texts
of this volumen. To help you develop your own perspective and a critical
attitude towards this work, you will find guiding questions and
suggestions that might prompt you to reflect on what you read throughout
the book. You'll recognize these prompts by the preceding {\#yourturn.}
Here is an example of what these questions look like:

\begin{quote}
{\#yourturn}

Do you think you might find such questions for reflection useful?
\end{quote}

Finally, we have enabled the option to collaboratively annotate this
work using \href{https://web.hypothes.is/}{hypothesis} (note that this
is how links are formatted in this book). Your annotations will be
visible to others, and others will be able to see yours, so that we can
build a better learning experience using this book together.

To read up on the original research we cite in this book, such as from
Vazire (\citeproc{ref-vazire2018}{2018}), you can hover over or click on
the references provided.

Feel free to make use of the resources in this book as you see fit. Our
hope is that they will support you in building a well-reflected opinion
about the existing body of knowledge in social psychology.

\bookmarksetup{startatroot}

\chapter*{\texorpdfstring{{Introduction}}{Introduction}}\label{introduction}
\addcontentsline{toc}{chapter}{{Introduction}}

\markboth{{Introduction}}{{Introduction}}

\section*{The Role of Change for Scientific
Discovery}\label{the-role-of-change-for-scientific-discovery}
\addcontentsline{toc}{section}{The Role of Change for Scientific
Discovery}

\markright{The Role of Change for Scientific Discovery}

Much of science capitalizes on change. It is the engine that drives
progress and the expansion of knowledge (see
\citeproc{ref-kuhn1962structure}{Kuhn 1962};
\citeproc{ref-popper1959logic}{Popper 1959}). Embracing change means
taking established theories and challenging them to explore new
directions. Changing perspectives, questioning the status quo, refining
existing concepts, and adapting to new evidence provide the stuff that
makes breakthroughs or new insights. In essence, change in science
represents taking steps forward, toward greater insight and reality
checks for the challenges we face. In other words, to push the
boundaries of what we know, we must make change.

\begin{quote}
{\#yourturn}

What instance of change regarding science have you recently heard about?
Consider reports of breakthroughs you might have seen in the news or
stories you saw on social media.
\end{quote}

In the past decade, Open Science has made change, by transforming
research practices to promote transparency, reproducibility, and
collaboration in scientific endeavors. By fostering a culture of
openness and collaboration, Open Science has brought about a paradigm
shift in research methodologies, paving the way for more robust and
reliable scientific discoveries
(\citeproc{ref-munafo2017manifesto}{Munafò et al. 2017};
\citeproc{ref-vazire2022credibility}{Vazire, Schiavone, and Bottesini
2022}). It is certainly no small feat to fundamentally reform how
research is done, and yet we have seen significant change towards Open
practices (\citeproc{ref-kidwell2016badges}{Kidwell et al. 2016};
\citeproc{ref-chambers2019registered}{Chambers 2019};
\citeproc{ref-christensen2020open}{Christensen et al. 2020}).

\begin{quote}
{\#definition} Open Science

An overhead term for a number of practices to make research more
transparent, such as making the data a research is project is based on
available to others.
\end{quote}

Change often implies the potential for a changed perception of what used
to be, particularly in comparison to what is now.

\begin{quote}
{\#yourturn}

Consider
\end{quote}

\bookmarksetup{startatroot}

\chapter*{\texorpdfstring{{Summary}}{Summary}}\label{summary}
\addcontentsline{toc}{chapter}{{Summary}}

\markboth{{Summary}}{{Summary}}

Add here

\section*{Take-Aways}\label{take-aways}
\addcontentsline{toc}{section}{Take-Aways}

\markright{Take-Aways}

Add here

\section*{Thanks}\label{thanks}
\addcontentsline{toc}{section}{Thanks}

\markright{Thanks}

This book was made possible by the many helping hands and critical
thoughts of the student authors involved in writing the individual
chapters. In addition, Melissa Engelbarth's support with selecting and
translating the chapters to include was invaluable.

\bookmarksetup{startatroot}

\chapter*{References}\label{references}
\addcontentsline{toc}{chapter}{References}

\markboth{References}{References}

\phantomsection\label{refs}
\begin{CSLReferences}{1}{0}
\bibitem[\citeproctext]{ref-chambers2019registered}
Chambers, Chris. 2019. {``The Registered Reports Revolution Lessons in
Cultural Reform.''} \emph{Significance} 16 (4): 23--27.

\bibitem[\citeproctext]{ref-christensen2020open}
Christensen, Garret, Zenan Wang, Elizabeth Levy Paluck, Nicholas
Swanson, David Birke, Edward Miguel, and Rebecca Littman. 2020. {``Open
Science Practices Are on the Rise: The State of Social Science (3S)
Survey.''}

\bibitem[\citeproctext]{ref-kidwell2016badges}
Kidwell, Mallory C, Ljiljana B Lazarević, Erica Baranski, Tom E
Hardwicke, Sarah Piechowski, Lina-Sophia Falkenberg, Curtis Kennett, et
al. 2016. {``Badges to Acknowledge Open Practices: A Simple, Low-Cost,
Effective Method for Increasing Transparency.''} \emph{PLoS Biology} 14
(5): e1002456.

\bibitem[\citeproctext]{ref-kuhn1962structure}
Kuhn, Thomas. 1962. {``The Structure of Scientific Revolutions.''}
\emph{International Encyclopedia of Unified Science} 2 (2).

\bibitem[\citeproctext]{ref-munafo2017manifesto}
Munafò, Marcus R, Brian A Nosek, Dorothy VM Bishop, Katherine S Button,
Christopher D Chambers, Nathalie Percie du Sert, Uri Simonsohn, Eric-Jan
Wagenmakers, Jennifer J Ware, and John Ioannidis. 2017. {``A Manifesto
for Reproducible Science.''} \emph{Nature Human Behaviour} 1 (1): 1--9.

\bibitem[\citeproctext]{ref-popper1959logic}
Popper, Karl R. 1959. \emph{The Logic of Scientific Discovery}.
Hutchinson \& Co.

\bibitem[\citeproctext]{ref-vazire2018}
Vazire, Simine. 2018. {``Implications of the Credibility Revolution for
Productivity, Creativity, and Progress.''} \emph{Perspectives on
Psychological Science} 13 (4): 411--17.
\url{https://doi.org/10.1177/1745691617751884}.

\bibitem[\citeproctext]{ref-vazire2022credibility}
Vazire, Simine, Sarah R Schiavone, and Julia G Bottesini. 2022.
{``Credibility Beyond Replicability: Improving the Four Validities in
Psychological Science.''} \emph{Current Directions in Psychological
Science} 31 (2): 162--68.

\end{CSLReferences}

\bookmarksetup{startatroot}

\chapter*{\texorpdfstring{{Glossary}}{Glossary}}\label{glossary}
\addcontentsline{toc}{chapter}{{Glossary}}

\markboth{{Glossary}}{{Glossary}}

Add here


\backmatter

\end{document}
